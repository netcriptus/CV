%%%%%%%%%%%%%%%%%%%%%%%%%%%%%%%%%%%%%%%%%%%%%%%%%%%%%%%%%%%%%%%%%%%%%%%%
%%%%%%%%%%%%%%%%%%%%%% Simple LaTeX CV Template %%%%%%%%%%%%%%%%%%%%%%%%
%%%%%%%%%%%%%%%%%%%%%%%%%%%%%%%%%%%%%%%%%%%%%%%%%%%%%%%%%%%%%%%%%%%%%%%%

%%%%%%%%%%%%%%%%%%%%%%%%%%%%%%%%%%%%%%%%%%%%%%%%%%%%%%%%%%%%%%%%%%%%%%%%
%% NOTE: If you find that it says                                     %%
%%                                                                    %%
%%                           1 of ??                                  %%
%%                                                                    %%
%% at the bottom of your first page, this means that the AUX file     %%
%% was not available when you ran LaTeX on this source. Simply RERUN  %% 
%% LaTeX to get the ``??'' replaced with the number of the last page  %% 
%% of the document. The AUX file will be generated on the first run   %%
%% of LaTeX and used on the second run to fill in all of the          %%
%% references.                                                        %%
%%%%%%%%%%%%%%%%%%%%%%%%%%%%%%%%%%%%%%%%%%%%%%%%%%%%%%%%%%%%%%%%%%%%%%%%

%%%%%%%%%%%%%%%%%%%%%%%%%%%% Document Setup %%%%%%%%%%%%%%%%%%%%%%%%%%%%

% Don't like 10pt? Try 11pt or 12pt
\documentclass[10pt]{article}

% This is a helpful package that puts math inside length specifications
\usepackage{calc}

% Layout: Puts the section titles on left side of page
\reversemarginpar

%
%         PAPER SIZE, PAGE NUMBER, AND DOCUMENT LAYOUT NOTES:
%
% The next \usepackage line changes the layout for CV style section
% headings as marginal notes. It also sets up the paper size as either
% letter or A4. By default, letter was used. If A4 paper is desired,
% comment out the letterpaper lines and uncomment the a4paper lines.
%
% As you can see, the margin widths and section title widths can be
% easily adjusted.
%
% ALSO: Notice that the includefoot option can be commented OUT in order
% to put the PAGE NUMBER *IN* the bottom margin. This will make the
% effective text area larger.
%
% IF YOU WISH TO REMOVE THE ``of LASTPAGE'' next to each page number,
% see the note about the +LP and -LP lines below. Comment out the +LP
% and uncomment the -LP.
%
% IF YOU WISH TO REMOVE PAGE NUMBERS, be sure that the includefoot line
% is uncommented and ALSO uncomment the \pagestyle{empty} a few lines
% below.
%

%% Use these lines for letter-sized paper
%\usepackage[paper=letterpaper,
%            %includefoot, % Uncomment to put page number above margin
%            marginparwidth=1.2in,     % Length of section titles
%            marginparsep=.05in,       % Space between titles and text
%            margin=1in,               % 1 inch margins
%            includemp]{geometry}

%% Use these lines for A4-sized paper
\usepackage[paper=a4paper,
%            %includefoot, % Uncomment to put page number above margin
            marginparwidth=30.5mm,    % Length of section titles
            marginparsep=1.5mm,       % Space between titles and text
            margin=25mm,              % 25mm margins
            includemp]{geometry}

%% More layout: Get rid of indenting throughout entire document
\setlength{\parindent}{0in}

%% This gives us fun enumeration environments. compactitem will be nice.
\usepackage{paralist}

%% Reference the last page in the page number
%
% NOTE: comment the +LP line and uncomment the -LP line to have page
%       numbers without the ``of ##'' last page reference)
%
% NOTE: uncomment the \pagestyle{empty} line to get rid of all page
%       numbers (make sure includefoot is commented out above)
%
\usepackage{fancyhdr,lastpage}
\pagestyle{fancy}
\pagestyle{empty}      % Uncomment this to get rid of page numbers
\fancyhf{}\renewcommand{\headrulewidth}{0pt}
\fancyfootoffset{\marginparsep+\marginparwidth}
\newlength{\footpageshift}
\setlength{\footpageshift}
          {0.5\textwidth+0.5\marginparsep+0.5\marginparwidth-2in}
\lfoot{\hspace{\footpageshift}%
       \parbox{4in}{\, \hfill %
                    \arabic{page} of \protect\pageref*{LastPage} % +LP
%                    \arabic{page}                               % -LP
                    \hfill \,}}

% Finally, give us PDF bookmarks
\usepackage{color,hyperref}
\definecolor{darkblue}{rgb}{0.0,0.0,0.3}
\hypersetup{colorlinks,breaklinks,
            linkcolor=darkblue,urlcolor=darkblue,
            anchorcolor=darkblue,citecolor=darkblue}

%%%%%%%%%%%%%%%%%%%%%%%% End Document Setup %%%%%%%%%%%%%%%%%%%%%%%%%%%%


%%%%%%%%%%%%%%%%%%%%%%%%%%% Helper Commands %%%%%%%%%%%%%%%%%%%%%%%%%%%%

% The title (name) with a horizontal rule under it
%
% Usage: \makeheading{name}
%
% Place at top of document. It should be the first thing.
\newcommand{\makeheading}[1]%
        {\hspace*{-\marginparsep minus \marginparwidth}%
         \begin{minipage}[t]{\textwidth+\marginparwidth+\marginparsep}%
                {\large \bfseries #1}\\[-0.15\baselineskip]%
                 \rule{\columnwidth}{1pt}%
         \end{minipage}}

% The section headings
%
% Usage: \section{section name}
%
% Follow this section IMMEDIATELY with the first line of the section
% text. Do not put whitespace in between. That is, do this:
%
%       \section{My Information}
%       Here is my information.
%
% and NOT this:
%
%       \section{My Information}
%
%       Here is my information.
%
% Otherwise the top of the section header will not line up with the top
% of the section. Of course, using a single comment character (%) on
% empty lines allows for the function of the first example with the
% readability of the second example.
\renewcommand{\section}[2]%
        {\pagebreak[2]\vspace{1.3\baselineskip}%
         \phantomsection\addcontentsline{toc}{section}{#1}%
         \hspace{0in}%
         \marginpar{
         \raggedright \scshape #1}#2}

% An itemize-style list with lots of space between items
\newenvironment{outerlist}[1][\enskip\textbullet]%
        {\begin{itemize}[#1]}{\end{itemize}%
         \vspace{-.6\baselineskip}}

% An environment IDENTICAL to outerlist that has better pre-list spacing
% when used as the first thing in a \section
\newenvironment{lonelist}[1][\enskip\textbullet]%
        {\vspace{-\baselineskip}\begin{list}{#1}{%
        \setlength{\partopsep}{0pt}%
        \setlength{\topsep}{0pt}}}
        {\end{list}\vspace{-.6\baselineskip}}

% An itemize-style list with little space between items
\newenvironment{innerlist}[1][\enskip\textbullet]%
        {\begin{compactitem}[#1]}{\end{compactitem}}

% To add some paragraph space between lines.
% This also tells LaTeX to preferably break a page on one of these gaps
% if there is a needed pagebreak nearby.
\newcommand{\blankline}{\quad\pagebreak[2]}

%%%%%%%%%%%%%%%%%%%%%%%% End Helper Commands %%%%%%%%%%%%%%%%%%%%%%%%%%%

%%%%%%%%%%%%%%%%%%%%%%%%% Begin CV Document %%%%%%%%%%%%%%%%%%%%%%%%%%%%

\begin{document}
\makeheading{Fernando Cezar Bernardelli}

\section{Contact Information}
%
% NOTE: Mind where the & separators and \\ breaks are in the following
%       table.
%
% ALSO: \rcollength is the width of the right column of the table 
%       (adjust it to your liking; default is 1.85in).
%
\newlength{\rcollength}\setlength{\rcollength}{1.85in}%
%
\begin{tabular}[t]{@{}p{\textwidth-\rcollength}p{\rcollength}}
2875, Algacyr Munhoz Mader Street     & \textit{Mobile:} +55 (41) 99776885 \\
Bl 06, Ap 03      & \textit{E-mail:}
\href{mailto:fernando@linhadefensiva.org}{fernando@linhadefensiva.org}\\
Curitiba, Parana, 81310-020, Brazil \\
\end{tabular}

\section{Citizenship}
%
Brazil

\section{Education}
%
\href{http://www.inf.ufpr.br/bcc}{\textbf{Federal University of
    Parana}}, Bachelor in Computer Science\\
Graduation date: July 2011

\href{http://www.inf.ufpr.br/pos}{\textbf{Federal University of
    Parana}}, Masters in Computer Science\\
Expected graduation date: July 2013


\section{Professional Experience}
%
\href{http://www.pet.inf.ufpr.br/}{\textbf{P.E.T. Research Group}},
Federal University of Parana
\begin{outerlist}

\item[] \textit{
        \hfill \textbf{March 2007 to February 2008}}
\begin{innerlist}
\item Ministrated Linux classes inside the University.
\item Researched projects using free, open source software.
\end{innerlist}

\end{outerlist}
\blankline

\textbf{LCG Software},
Curitiba, Parana, Brazil
\begin{outerlist}

\item[] \textit{Development of web bots and parsers}%
        \hfill \textbf{January 2008 to October 2009}
\begin{innerlist}
\item Designed and implemented parsers for sports betting websites.
\item Participated in software development of web bots for automated
  betting.
\end{innerlist}

\end{outerlist}

\blankline

\href{http://www.maestrosoft.com.br/}{\textbf{Maestro Softwares}},
Curitiba, Parana, Brazil
\begin{outerlist}

\item[] \textit{Development of web systems}%
        \hfill \textbf{October 2009 to July 2010}
\begin{innerlist}
\item Bug fixes and development of new functions in the control system
  of \href{http://www.genpro.com.br/}{\textbf{Genpro engineering}}
\item Implemented and configured a backup system (Bacula) for
  distributed servers.
\end{innerlist}

\end{outerlist}

\blankline

\textbf{Alian\c{c}a Project},
Curitiba, Parana, Brazil
\begin{outerlist}

\item[] \textit{Supervisor of web system development}%
        \hfill \textbf{July 2010 to July 2011}
\begin{innerlist}
\item Responsible for planning and development of a new web system for file sharing based on SMS.
\item Reponsible for selecting, hiring and coordinating the development team.
\end{innerlist}

\end{outerlist}

\blankline


\section{Technical Skills}\\
%
Over four years of professional and academic experience in Python.

Programming skills in Pascal, C, Shell Script, PHP, Javascript, Ruby on Rails and Ajax.

Experience with Bacula

Brief experience in network administration.

Skills as Linux sysadmin.

Fluency in English (Cambridge Certificated) and Portuguese native speaker.

Short experience with robots building and basics hardware design.


\section{Events}
\begin{itemize}
\item Attended to \href{http://www.fisl.org.br/10/www/}{\textbf{FISL
    (International Free Software Forum)}} in the years of 2008, 2009, 2010 and 2011.

\item Attended to a one week python classes in the year of 2008, at
\href{http://www.inf.ufpr.br/bcc}{\textbf{Federal University of
    Parana}}.

\item Attended Google Developer Day Brazil 2010 and 2011.
\end{itemize}


\section{Volunteer work}
\begin{itemize}
\item Ministrated Linux classes for children in local poor community.
\item Coordinator of moderation staff of
  \href{http://linhadefensiva.org}{\textbf{Linha Defensiva}}, an online
  IT forum in Brazil.
\item Ministrated Linux classes for students at Federal University of Parana
\end{itemize}

\end{document}

%%%%%%%%%%%%%%%%%%%%%%%%%% End CV Document %%%%%%%%%%%%%%%%%%%%%%%%%%%%%
