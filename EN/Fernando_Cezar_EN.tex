%%%%%%%%%%%%%%%%%%%%%%%%%%%%%%%%%%%%%%%%%%%%%%%%%%%%%%%%%%%%%%%%%%%%%%%%
%%%%%%%%%%%%%%%%%%%%%% Simple LaTeX CV Template %%%%%%%%%%%%%%%%%%%%%%%%
%%%%%%%%%%%%%%%%%%%%%%%%%%%%%%%%%%%%%%%%%%%%%%%%%%%%%%%%%%%%%%%%%%%%%%%%

%%%%%%%%%%%%%%%%%%%%%%%%%%%%%%%%%%%%%%%%%%%%%%%%%%%%%%%%%%%%%%%%%%%%%%%%
%% NOTE: If you find that it says                                     %%
%%                                                                    %%
%%                           1 of ??                                  %%
%%                                                                    %%
%% at the bottom of your first page, this means that the AUX file     %%
%% was not available when you ran LaTeX on this source. Simply RERUN  %%
%% LaTeX to get the ``??'' replaced with the number of the last page  %%
%% of the document. The AUX file will be generated on the first run   %%
%% of LaTeX and used on the second run to fill in all of the          %%
%% references.                                                        %%
%%%%%%%%%%%%%%%%%%%%%%%%%%%%%%%%%%%%%%%%%%%%%%%%%%%%%%%%%%%%%%%%%%%%%%%%

%%%%%%%%%%%%%%%%%%%%%%%%%%%% Document Setup %%%%%%%%%%%%%%%%%%%%%%%%%%%%

% Don't like 10pt? Try 11pt or 12pt
\documentclass[10pt]{article}

% This is a helpful package that puts math inside length specifications
\usepackage{calc}

% Layout: Puts the section titles on left side of page
\reversemarginpar

%
%         PAPER SIZE, PAGE NUMBER, AND DOCUMENT LAYOUT NOTES:
%
% The next \usepackage line changes the layout for CV style section
% headings as marginal notes. It also sets up the paper size as either
% letter or A4. By default, letter was used. If A4 paper is desired,
% comment out the letterpaper lines and uncomment the a4paper lines.
%
% As you can see, the margin widths and section title widths can be
% easily adjusted.
%
% ALSO: Notice that the includefoot option can be commented OUT in order
% to put the PAGE NUMBER *IN* the bottom margin. This will make the
% effective text area larger.
%
% IF YOU WISH TO REMOVE THE ``of LASTPAGE'' next to each page number,
% see the note about the +LP and -LP lines below. Comment out the +LP
% and uncomment the -LP.
%
% IF YOU WISH TO REMOVE PAGE NUMBERS, be sure that the includefoot line
% is uncommented and ALSO uncomment the \pagestyle{empty} a few lines
% below.
%

%% Use these lines for letter-sized paper
%\usepackage[paper=letterpaper,
%            %includefoot, % Uncomment to put page number above margin
%            marginparwidth=1.2in,     % Length of section titles
%            marginparsep=.05in,       % Space between titles and text
%            margin=1in,               % 1 inch margins
%            includemp]{geometry}

%% Use these lines for A4-sized paper
\usepackage[paper=a4paper,
%            %includefoot, % Uncomment to put page number above margin
            marginparwidth=30.5mm,    % Length of section titles
            marginparsep=1.5mm,       % Space between titles and text
            margin=25mm,              % 25mm margins
            includemp]{geometry}

%% More layout: Get rid of indenting throughout entire document
\setlength{\parindent}{0in}

%% This gives us fun enumeration environments. compactitem will be nice.
\usepackage{paralist}

%% Reference the last page in the page number
%
% NOTE: comment the +LP line and uncomment the -LP line to have page
%       numbers without the ``of ##'' last page reference)
%
% NOTE: uncomment the \pagestyle{empty} line to get rid of all page
%       numbers (make sure includefoot is commented out above)
%
\usepackage{fancyhdr,lastpage}
\pagestyle{fancy}
\pagestyle{empty}      % Uncomment this to get rid of page numbers
\fancyhf{}\renewcommand{\headrulewidth}{0pt}
\fancyfootoffset{\marginparsep+\marginparwidth}
\newlength{\footpageshift}
\setlength{\footpageshift}
          {0.5\textwidth+0.5\marginparsep+0.5\marginparwidth-2in}
\lfoot{\hspace{\footpageshift}%
       \parbox{4in}{\, \hfill %
                    \arabic{page} of \protect\pageref*{LastPage} % +LP
%                    \arabic{page}                               % -LP
                    \hfill \,}}

% Finally, give us PDF bookmarks
\usepackage{color,hyperref}
\definecolor{darkblue}{rgb}{0.0,0.0,0.3}
\hypersetup{colorlinks,breaklinks,
            linkcolor=darkblue,urlcolor=darkblue,
            anchorcolor=darkblue,citecolor=darkblue}

%%%%%%%%%%%%%%%%%%%%%%%% End Document Setup %%%%%%%%%%%%%%%%%%%%%%%%%%%%


%%%%%%%%%%%%%%%%%%%%%%%%%%% Helper Commands %%%%%%%%%%%%%%%%%%%%%%%%%%%%

% The title (name) with a horizontal rule under it
%
% Usage: \makeheading{name}
%
% Place at top of document. It should be the first thing.
\newcommand{\makeheading}[1]%
        {\hspace*{-\marginparsep minus \marginparwidth}%
         \begin{minipage}[t]{\textwidth+\marginparwidth+\marginparsep}%
                {\large \bfseries #1}\\[-0.15\baselineskip]%
                 \rule{\columnwidth}{1pt}%
         \end{minipage}}

% The section headings
%
% Usage: \section{section name}
%
% Follow this section IMMEDIATELY with the first line of the section
% text. Do not put whitespace in between. That is, do this:
%
%       \section{My Information}
%       Here is my information.
%
% and NOT this:
%
%       \section{My Information}
%
%       Here is my information.
%
% Otherwise the top of the section header will not line up with the top
% of the section. Of course, using a single comment character (%) on
% empty lines allows for the function of the first example with the
% readability of the second example.
\renewcommand{\section}[2]%
        {\pagebreak[2]\vspace{1.3\baselineskip}%
         \phantomsection\addcontentsline{toc}{section}{#1}%
         \hspace{0in}%
         \marginpar{
         \raggedright \scshape #1}#2}

% An itemize-style list with lots of space between items
\newenvironment{outerlist}[1][\enskip\textbullet]%
        {\begin{itemize}[#1]}{\end{itemize}%
         \vspace{-.6\baselineskip}}

% An environment IDENTICAL to outerlist that has better pre-list spacing
% when used as the first thing in a \section
\newenvironment{lonelist}[1][\enskip\textbullet]%
        {\vspace{-\baselineskip}\begin{list}{#1}{%
        \setlength{\partopsep}{0pt}%
        \setlength{\topsep}{0pt}}}
        {\end{list}\vspace{-.6\baselineskip}}

% An itemize-style list with little space between items
\newenvironment{innerlist}[1][\enskip\textbullet]%
        {\begin{compactitem}[#1]}{\end{compactitem}}

% To add some paragraph space between lines.
% This also tells LaTeX to preferably break a page on one of these gaps
% if there is a needed pagebreak nearby.
\newcommand{\blankline}{\quad\pagebreak[2]}

%%%%%%%%%%%%%%%%%%%%%%%% End Helper Commands %%%%%%%%%%%%%%%%%%%%%%%%%%%

%%%%%%%%%%%%%%%%%%%%%%%%% Begin CV Document %%%%%%%%%%%%%%%%%%%%%%%%%%%%

\begin{document}
\makeheading{Fernando Cezar Bernardelli}

\section{Contact Information}
%
% NOTE: Mind where the & separators and \\ breaks are in the following
%       table.
%
% ALSO: \rcollength is the width of the right column of the table
%       (adjust it to your liking; default is 1.85in).
%
\newlength{\rcollength}\setlength{\rcollength}{1.85in}%
%
\begin{tabular}[t]{@{}p{\textwidth-\rcollength}p{\rcollength}}
Gaillardstra{\ss}e, 20a     & \textit{Mobile:} +49 152 0640 1548 \\
Berlin, 13187, Germany & \textit{E-mail:}
\href{mailto:fernando@cezar.link}{fernando@cezar.link}\\
\end{tabular}

\section{Citizenship}
%
Brazil

\section{Education}
%
\href{http://www.inf.ufpr.br/bcc}{\textbf{Federal University of
    Parana}}, Bachelor in Computer Science\\
Graduation date: July 2011\\
\\
\href{http://www.inf.ufpr.br/bcc}{\textbf{Federal University of
    Parana}}, Masters in Computer Science -- with emphasis on advanced networks and security\\
From July 2011 to January 2013 -- Unfinished.

\blankline

\section{Public links}

\textbf{Github}: \href{https://github.com/netcriptus}{https://github.com/netcriptus}
\\
\textbf{LinkedIn}: \href{https://de.linkedin.com/in/fernando-cezar-bernardelli-8b333a32}{https://de.linkedin.com/in/fernando-cezar-bernardelli-8b333a32}\\
\textbf{StackOverflow}: \href{https://stackoverflow.com/users/1619435/fernando-cezar}{https://stackoverflow.com/users/1619435/fernando-cezar}

\blankline

\section{Professional Experience}
%
\href{https://www.brainbot.com/}{\textbf{BrainBot Technologies}},
Berlin, Germany
\begin{outerlist}

\item[] \textit{Principal Software Engineer}%
        \hfill \textbf{March 2021 to present}
\begin{innerlist}
    \item Part of the team responsible for development of the \href{https://github.com/raiden-network/raiden}{Raiden Protocol}. The main responsibilities include writing and maintaining the \href{https://github.com/raiden-network/raiden-contracts}{Smart Contracts} on the Ethereum blockchain, as well as the \href{https://github.com/raiden-network/raiden-services}{Raiden Ecosystem} necessary for the open source project.
    \item Project lead for the integration of Raiden with the FetchAI framework.
    \item Technical manager and first point of contact for community technical support on the \href{https://discord.com/invite/nSQDQBq5FC}{Raiden Discord server}.
    \item Technologies used for the company backend include Python, Flask and Solidity.
\end{innerlist}

\end{outerlist}

\blankline

\href{http://planet.com/}{\textbf{Planet Labs}},
Berlin, Germany
\begin{outerlist}

\item[] \textit{Senior Backend Developer}%
        \hfill \textbf{January 2017 to March 2021}
\begin{innerlist}
    \item Responsible for architecting and maintaining many systems related
     to access control, account managing and satellite tasking.
    \item Occasionally acted as technical team lead when the situation demanded.
    \item Technologies used for the company backend include Python, Django, Flask and Golang.
\end{innerlist}

\end{outerlist}

\blankline

\href{http://dubsmash.com/}{\textbf{Dubsmash}},
Berlin, Germany
\begin{outerlist}

\item[] \textit{Senior Backend Developer}%
        \hfill \textbf{February 2016 to January 2017}
\begin{innerlist}
    \item API architect and developer for microservices.
    \item Technologies used in the company backend include Python, Django, Golang, PostgreSQL, DynamoDB,
     Amazon Web Services, Celery, RabbitMQ, git and docker.
    \item Methods used in the company include Scrum, Kanban and TDD.
\end{innerlist}

\end{outerlist}

\blankline

\href{http://eatfirst.com/}{\textbf{EatFirst}},
Berlin, Germany
\begin{outerlist}

\item[] \textit{Senior Full Stack Developer}%
        \hfill \textbf{April 2015 to February 2016}
\begin{innerlist}
    \item API developer for an e-commerce platform, built from scratch.
    \item Technologies used in the company for backend include Python, Flask, PostgreSQL, Amazon Web Services
    (S2 and E3 mainly), git, Nginx and docker.
    \item Stack for frontend include ReactJS, Backbone and RiotJS.
    \item Methods used in the company include Scrum and TDD.
\end{innerlist}

\end{outerlist}

\blankline

\href{http://titansgroup.com.br/}{\textbf{Titans Group}},
Sao Paulo, SP, Brazil
\begin{outerlist}

\item[] \textit{Backend Developer}%
        \hfill \textbf{October 2012 to March 2015}
\begin{innerlist}
    \item API developer for whitelabel products, including:
        \begin{innerlist}
            \item a file sync software
            \item a Contacts management WebDAV compliant server
            \item a Single Sign On software and federated environment.
        \end{innerlist}
    \item Occasionally performed pentests in the company's products.

    \item Technologies used in the company include Python, Django, Flask, Shell Script, MySQL, PostgreSQL, Amazon Web Services
    (S2 and E3), git, Nginx and MongoDB.
    \item Methods and tools used in the company include Scrum, TDD, BDD, Mumble and Grove.io (both for communication
    with inner and remote team)
\end{innerlist}

\end{outerlist}

\blankline

\href{http://www.muccashop.com.br/}{\textbf{Mucca Shop}},
Curitiba, Parana, Brazil
\begin{outerlist}

\item[] \textit{Backend Developer}%
        \hfill \textbf{July 2012 to October 2012}
\begin{innerlist}
    \item Part of the team developing \href{http://www.orelhadelivro.com.br/}{Orelha de Livro}, a web
    app for books management.
    \item Technologies used by the company include Amazon Web Services (S3 and RDS), git, Python, Django and MySQL
\end{innerlist}

\end{outerlist}

\blankline

\textbf{LCG Software},
Curitiba, Parana, Brazil
\begin{outerlist}

\item[] \textit{Development of web bots and parsers, intern}%
        \hfill \textbf{January 2008 to October 2009}
\begin{innerlist}
    \item Designed and implemented html parsers for sports betting websites.
    \item Participated in software development of web bots for automated
      betting.
    \item Technologies used included Python, Twisted and PAMIE.
\end{innerlist}

\end{outerlist}

\blankline

\href{http://pet.inf.ufpr.br/}{\textbf{P.E.T. Research Group}},
Federal University of Parana
\begin{outerlist}

\item[] \textit{
        \hfill \textbf{March 2007 to February 2008}}
\begin{innerlist}
    \item Ministrated Linux classes inside the University.
    \item Researched projects using free, open source software, highlights on:
    \begin{innerlist}
        \item Dokeos and Moodle, deployed and compared on a research about e-learning
        \item ffmpeg for real time video streaming conversion
    \end{innerlist}
    \item Technologies adopted by the group include C, PHP, Shell Script, Subversion, Python and Django
\end{innerlist}

\end{outerlist}
\blankline

\section{Technical Skills}\\
%
Over 15 years of professional and academic experience with the Python language.

Programming skills in Ruby, C/C++, Shell Script, Solidity, Javascript and Go.

Extensive experience in the web frameworks Django and Flask, and considerable experience with Rails and Sinatra.

Familiar with methods such as BDD, TDD and Scrum.

Academic research about advanced mobile networks (mainly MANETs and DTN) and network security.

Skilled with \LaTeX


\section{Events}
\begin{itemize}
\item Main speaker at the \href{https://www.meetup.com/golang-users-berlin/events/232442778/}
{Golang User Group Berlin 5th anniversary meetup}

\item Second place award in the \href{https://devpost.com/software/roboteyes}{Hack HPI Hackathon 2016}

\end{itemize}

\end{document}

%%%%%%%%%%%%%%%%%%%%%%%%%% End CV Document %%%%%%%%%%%%%%%%%%%%%%%%%%%%%
