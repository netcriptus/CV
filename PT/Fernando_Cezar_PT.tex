%%%%%%%%%%%%%%%%%%%%%%%%%%%% Document Setup %%%%%%%%%%%%%%%%%%%%%%%%%%%%

% Don't like 10pt? Try 11pt or 12pt
\documentclass[10pt]{article}

% This is a helpful package that puts math inside length specifications
\usepackage{calc}

%This is needed for the portuguese version
\usepackage[utf8]{inputenc}

% Layout: Puts the section titles on left side of page
\reversemarginpar

%
%         PAPER SIZE, PAGE NUMBER, AND DOCUMENT LAYOUT NOTES:
%
% The next \usepackage line changes the layout for CV style section
% headings as marginal notes. It also sets up the paper size as either
% letter or A4. By default, letter was used. If A4 paper is desired,
% comment out the letterpaper lines and uncomment the a4paper lines.
%
% As you can see, the margin widths and section title widths can be
% easily adjusted.
%
% ALSO: Notice that the includefoot option can be commented OUT in order
% to put the PAGE NUMBER *IN* the bottom margin. This will make the
% effective text area larger.
%
% IF YOU WISH TO REMOVE THE ``of LASTPAGE'' next to each page number,
% see the note about the +LP and -LP lines below. Comment out the +LP
% and uncomment the -LP.
%
% IF YOU WISH TO REMOVE PAGE NUMBERS, be sure that the includefoot line
% is uncommented and ALSO uncomment the \pagestyle{empty} a few lines
% below.
%

%% Use these lines for letter-sized paper
%\usepackage[paper=letterpaper,
%            %includefoot, % Uncomment to put page number above margin
%            marginparwidth=1.2in,     % Length of section titles
%            marginparsep=.05in,       % Space between titles and text
%            margin=1in,               % 1 inch margins
%            includemp]{geometry}

%% Use these lines for A4-sized paper
\usepackage[paper=a4paper,
%            %includefoot, % Uncomment to put page number above margin
            marginparwidth=30.5mm,    % Length of section titles
            marginparsep=1.5mm,       % Space between titles and text
            margin=25mm,              % 25mm margins
            includemp]{geometry}

%% More layout: Get rid of indenting throughout entire document
\setlength{\parindent}{0in}

%% This gives us fun enumeration environments. compactitem will be nice.
\usepackage{paralist}

%% Reference the last page in the page number
%
% NOTE: comment the +LP line and uncomment the -LP line to have page
%       numbers without the ``of ##'' last page reference)
%
% NOTE: uncomment the \pagestyle{empty} line to get rid of all page
%       numbers (make sure includefoot is commented out above)
%
\usepackage{fancyhdr,lastpage}
\pagestyle{fancy}
\pagestyle{empty}      % Uncomment this to get rid of page numbers
\fancyhf{}\renewcommand{\headrulewidth}{0pt}
\fancyfootoffset{\marginparsep+\marginparwidth}
\newlength{\footpageshift}
\setlength{\footpageshift}
          {0.5\textwidth+0.5\marginparsep+0.5\marginparwidth-2in}
\lfoot{\hspace{\footpageshift}%
       \parbox{4in}{\, \hfill %
                    \arabic{page} of \protect\pageref*{LastPage} % +LP
%                    \arabic{page}                               % -LP
                    \hfill \,}}

% Finally, give us PDF bookmarks
\usepackage{color,hyperref}
\definecolor{darkblue}{rgb}{0.0,0.0,0.3}
\hypersetup{colorlinks,breaklinks,
            linkcolor=darkblue,urlcolor=darkblue,
            anchorcolor=darkblue,citecolor=darkblue}

%%%%%%%%%%%%%%%%%%%%%%%% End Document Setup %%%%%%%%%%%%%%%%%%%%%%%%%%%%


%%%%%%%%%%%%%%%%%%%%%%%%%%% Helper Commands %%%%%%%%%%%%%%%%%%%%%%%%%%%%

% The title (name) with a horizontal rule under it
%
% Usage: \makeheading{name}
%
% Place at top of document. It should be the first thing.
\newcommand{\makeheading}[1]%
        {\hspace*{-\marginparsep minus \marginparwidth}%
         \begin{minipage}[t]{\textwidth+\marginparwidth+\marginparsep}%
                {\large \bfseries #1}\\[-0.15\baselineskip]%
                 \rule{\columnwidth}{1pt}%
         \end{minipage}}

% The section headings
%
% Usage: \section{section name}
%
% Follow this section IMMEDIATELY with the first line of the section
% text. Do not put whitespace in between. That is, do this:
%
%       \section{My Information}
%       Here is my information.
%
% and NOT this:
%
%       \section{My Information}
%
%       Here is my information.
%
% Otherwise the top of the section header will not line up with the top
% of the section. Of course, using a single comment character (%) on
% empty lines allows for the function of the first example with the
% readability of the second example.
\renewcommand{\section}[2]%
        {\pagebreak[2]\vspace{1.3\baselineskip}%
         \phantomsection\addcontentsline{toc}{section}{#1}%
         \hspace{0in}%
         \marginpar{
         \raggedright \scshape #1}#2}

% An itemize-style list with lots of space between items
\newenvironment{outerlist}[1][\enskip\textbullet]%
        {\begin{itemize}[#1]}{\end{itemize}%
         \vspace{-.6\baselineskip}}

% An environment IDENTICAL to outerlist that has better pre-list spacing
% when used as the first thing in a \section
\newenvironment{lonelist}[1][\enskip\textbullet]%
        {\vspace{-\baselineskip}\begin{list}{#1}{%
        \setlength{\partopsep}{0pt}%
        \setlength{\topsep}{0pt}}}
        {\end{list}\vspace{-.6\baselineskip}}

% An itemize-style list with little space between items
\newenvironment{innerlist}[1][\enskip\textbullet]%
        {\begin{compactitem}[#1]}{\end{compactitem}}

% To add some paragraph space between lines.
% This also tells LaTeX to preferably break a page on one of these gaps
% if there is a needed pagebreak nearby.
\newcommand{\blankline}{\quad\pagebreak[2]}

%%%%%%%%%%%%%%%%%%%%%%%% End Helper Commands %%%%%%%%%%%%%%%%%%%%%%%%%%%

%%%%%%%%%%%%%%%%%%%%%%%%% Begin CV Document %%%%%%%%%%%%%%%%%%%%%%%%%%%%

\begin{document}
\makeheading{Fernando Cezar Bernardelli}

\section{Contato}
%
% NOTE: Mind where the & separators and \\ breaks are in the following
%       table.
%
% ALSO: \rcollength is the width of the right column of the table 
%       (adjust it to your liking; default is 1.85in).
%
\newlength{\rcollength}\setlength{\rcollength}{1.85in}%
%
\begin{tabular}[t]{@{}p{\textwidth-\rcollength}p{\rcollength}}
\textit{Telefone:} +55 (41) 9977-6885 & Algacyr Munhoz Mader, 2875 \\
\textit{E-mail:}
\href{mailto:fernando@linhadefensiva.org}{fernando@linhadefensiva.org}
                                      & Bl 06, Ap 03 \\
                                      & Curitiba, Paraná \\
\end{tabular}

\section{Formação}
%
\href{http://www.inf.ufpr.br/bcc}{\textbf{Universidade Federal do
    Paraná}}, Bacharelado em Ciência da Computação\\
Data prevista para conclusão: Junho de 2011

\section{Experiência Profissional}
%
\href{http://www.pet.inf.ufpr.br/}{\textbf{Grupo PET}}(Programa de
Educação Tutorial),
Universidade Federal do Paraná
\begin{outerlist}

\item[] \textit{
        \hfill \textbf{Março de 2007 a Fevereiro de 2008}}
\begin{innerlist}
\item Ministrou aulas de Linux para estudantes da UFPR.
\item Pesquisas em projetos Open Source.
\end{innerlist}

\end{outerlist}
\blankline

\textbf{LCG Software},
Curitiba, Paraná, Brazil
\begin{outerlist}

\item[] \textit{Desenvolvimento de bots e parsers}%
        \hfill \textbf{Janeiro de 2008 a Outubro de 2009}
\begin{innerlist}
\item Desenvolvimento e implementação de parsers para sites de apostas
  em esportes.
\item Desenvolvimento e implementação de bots para automatização.
\end{innerlist}

\end{outerlist}

\blankline

\textbf{Maestro Softwares},
Curitiba, Paraná, Brazil
\begin{outerlist}

\item[] \textit{Desenvolvimento e Sysadmin}%
        \hfill \textbf{Outubro de 2009 a Julho de 2010}
\begin{innerlist}
\item Desenvolvimento e correção de bugs no sistema de gerenciamento de
  projetos da \href{http://www.genpro.com.br}{\textbf{Genpro}}.
\item Administração do sistema de backup usando Bacula.
\end{innerlist}

\end{outerlist}

\blankline

\textbf{Aliança Project},
Curitiba, Paraná, Brazil
\begin{outerlist}

\item[] \textit{Supervisor, gerente de projeto e desenvolvedor}%
        \hfill \textbf{Julho de 2010 a Julho de 2011}
\begin{innerlist}
\item Responsável pela seleção e coordenação da equipe de desenvolvimento de projetos voltados a web, usando Ruby on Rails.
\end{innerlist}

\end{outerlist}

\blankline


\section{Eventos}
\begin{itemize}
\item Participante do \href{http://www.fisl.org.br/10/www/}{\textbf{FISL
    (Fórum Internacional de Software Livre)}} nos anos de 2008, 2009, 2010 e 2011.
    
\item Participante do \href{http://www.google.com/events/developerday/2010/}{\textbf{Google Developer Day Brasil}} edições de 2010 e 2011.

\item Participante do curso de python ofertado pela \href{http://www.inf.ufpr.br/bcc}{\textbf{UFPR}} no ano de 2008.

\item Participante da primeira edição da conferência de segurança da informação \href{http://www.sbconference.com.br/}{\textbf{Silver Bullet}}, em 2011, São Paulo.
\end{itemize}

\blankline

\section{Qualificações técnicas}
%
\begin{itemize}
  \item Mais de 5 anos de experiência profissional e acadêmica em Python.
  \blankline
  \item Experiência de programação em Pascal, C, Shell Script, Ruby on Rails e PHP.
  \blankline
  \item Pequena experiência em administração de redes.
  \blankline
  \item Experiência em administração de sistemas Linux.
  \blankline
  \item Breve experiência em administração de redes.
  \blankline
  \item Inglês fluente (Cambridge Certified).
  \blankline
  \item Pequena experiência em construção de Robôs e projeto de Hardware.
  \blankline
  \item Domínio razoável de LaTex.
  \blankline
  \item Experiência com Bacula.
\end{itemize}


\section{Trabalhos Voluntários}
%
\begin{itemize}
\item Aulas de Linux Básico na comunidade da Vila Torres (como
  voluntário do Projeto Negaça, pela UFPR).
\blankline
\item Membro da equipe de moderação do fórum
  \href{http://www.linhadefensiva.org}{\textbf{Linha Defensiva}}, um dos
  maiores fóruns online de informática do Brasil.

\end{itemize}

\section{Áreas de interesse}
%
\begin{itemize}
\item Desenvolvimento de sistemas desktop e voltados a web.
\blankline
\item Segurança, projeto e administração de redes Unix.
\blankline
\item Desenvolvimento de protocolos.
\blankline
\item Administração de servidores Unix.

\end{itemize}


\end{document}

%%%%%%%%%%%%%%%%%%%%%%%%%% End CV Document %%%%%%%%%%%%%%%%%%%%%%%%%%%%%
